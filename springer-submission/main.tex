%Version 3.1 December 2024
% Springer Nature LaTeX Template for Digital Finance submission
% Converted from Farzulla Research preprint v2.0.1

\documentclass[pdflatex,sn-mathphys-num]{sn-jnl}

%%%% Standard Packages
\usepackage{graphicx}
\usepackage{multirow}
\usepackage{amsmath,amssymb,amsfonts}
\usepackage{amsthm}
\usepackage{mathrsfs}
\usepackage[title]{appendix}
\usepackage{xcolor}
\usepackage{textcomp}
\usepackage{manyfoot}
\usepackage{booktabs}
\usepackage{enumitem}

% Figure path
\graphicspath{{publication_figures/}}

% Theorem styles
\theoremstyle{thmstyleone}
\newtheorem{theorem}{Theorem}
\newtheorem{proposition}[theorem]{Proposition}

\theoremstyle{thmstyletwo}
\newtheorem{example}{Example}
\newtheorem{remark}{Remark}

\theoremstyle{thmstylethree}
\newtheorem{definition}{Definition}

\raggedbottom

\begin{document}

\title[Infrastructure vs Regulatory Shocks in Crypto]{Infrastructure vs Regulatory Shocks: Asymmetric Volatility Response in Cryptocurrency Markets}

\author*[1,2]{\fnm{Murad} \sur{Farzulla}}\email{murad@farzulla.org}

\affil*[1]{\orgdiv{King's Business School}, \orgname{King's College London}, \orgaddress{\city{London}, \country{United Kingdom}}}

\affil[2]{\orgname{Farzulla Research}, \orgaddress{\city{London}, \country{United Kingdom}}}

%%==================================%%
%% Abstract - REFRAMED VERSION
%%==================================%%

\abstract{
Infrastructure failures generate 5.7$\times$ larger volatility shocks than regulatory announcements in cryptocurrency markets (2.385\% vs 0.419\%, $p=0.0008$, Cohen's $d=2.753$), challenging assumptions that ``all bad news is equivalent'' for portfolio risk management. This asymmetry is robust across six major cryptocurrencies (January 2019--August 2025), multiple statistical tests, and validation approaches including Bayesian inference (Bayes Factors $>$10 for 4/6 assets), machine learning clustering, network spillover analysis, and Markov regime-switching models.

We analyze 50 major events using GJR-GARCH-X models incorporating infrastructure disruptions (exchange outages, protocol exploits, network failures) and regulatory announcements (enforcement actions, policy changes) as exogenous variance drivers. A novel GDELT sentiment decomposition separates regulatory from infrastructure-related news coverage, enabling event-specific sentiment analysis.

Critically, even degraded sentiment proxies---weekly aggregation creating 7-day temporal mismatch with daily volatility, 7\% missing values, and systematic negative bias---improve model fit for 83\% of assets. This suggests sentiment's true information content is substantially \textit{underestimated} in our results: cryptocurrency markets appear sufficiently sentiment-driven that any reasonable proxy captures tradeable signal, implying higher-frequency sentiment data would yield considerably stronger effects.

Network analysis reveals ETH, not BTC, serves as the primary systemic risk hub (eigenvector centrality 0.89 vs 0.71), challenging conventional assumptions about Bitcoin dominance. Regime-switching models detect 5$\times$ sensitivity amplification during crisis periods ($F=45.23$, $p<0.001$), with infrastructure sensitivity increasing from 2.3\% to 11.2\% during market stress---implying traditional VaR models assuming linear risk scaling catastrophically underestimate tail risk.

Portfolio managers should allocate 4--5$\times$ higher capital buffers for infrastructure events. The near-integrated volatility persistence ($\alpha+\beta \approx 0.999$) suggests cryptocurrency markets operate in a distinct regime where shocks become absorbed into long-memory processes, posing fundamental challenges for traditional risk management frameworks.
}

\keywords{Cryptocurrency, Volatility, Event study, GJR-GARCH-X, Infrastructure risk, Regulatory uncertainty}

\maketitle

% ============================================================================
% INTRODUCTION
% ============================================================================

\section{Introduction}\label{sec:intro}

The cryptocurrency market's transformation from experimental technology to a multi-trillion-dollar asset class has created unprecedented challenges for understanding information processing in financial markets. Since Bitcoin's inception in 2009, digital assets have developed unique structural characteristics---continuous 24/7 trading, fragmented exchange infrastructure, predominantly retail participation, and critical technological dependencies---that fundamentally distinguish them from traditional financial markets \citep{MakarovSchoar2020, Urquhart2016}. These features violate core assumptions of classical market efficiency theory \citep{Fama1970} and necessitate new frameworks for understanding how different information types are processed and incorporated into prices.

A critical question for risk management is whether markets differentiate between event types, or whether ``all bad news is equivalent.'' Regulatory events---such as enforcement actions, policy changes, or legislative proposals---affect valuations through expectation channels, requiring investors to reassess fundamental value based on changing legal constraints \citep{AuerClaessens2018, FeinsteinWerbach2021}. Infrastructure failures---including exchange outages, protocol exploits, and network disruptions---create immediate mechanical disruptions to trading and settlement mechanisms \citep{ChenEtAl2023, Grobys2021}. Despite extensive research on each event type in isolation, no study has directly compared their volatility impacts within a unified econometric framework.

This study addresses these gaps through three contributions. First, we develop a unified GJR-GARCH-X framework that enables direct comparison of infrastructure and regulatory events within consistent econometric specifications. Second, we construct decomposed GDELT-based sentiment indices separating regulatory from infrastructure-related news coverage---a novel methodology enabling event-type-specific sentiment analysis. Third, we examine whether event sensitivity varies across market regimes, testing whether infrastructure shocks amplify during crisis periods when liquidity is already impaired.

We investigate three hypotheses:

\begin{description}[leftmargin=0pt,labelindent=0pt]
\item[H1: Asymmetric Event Impact.] Infrastructure events generate significantly larger volatility impacts than regulatory events, reflecting immediate mechanical disruption versus gradual information absorption.

\item[H2: Sentiment Explanatory Power.] Even degraded sentiment proxies---characterised by temporal aggregation, missing values, and measurement noise---provide incremental explanatory power for volatility dynamics beyond event dummies alone.

\item[H3: Crisis Amplification.] Infrastructure event sensitivity amplifies during crisis regimes, as mechanical disruptions compound with already-impaired market liquidity.
\end{description}

Through comprehensive analysis of 50 major events across six cryptocurrencies from January 2019 to August 2025, we find strong support for all three hypotheses. Infrastructure events generate volatility impacts 5.7 times larger than regulatory events (2.385\% vs 0.419\%, $p=0.0008$, Cohen's $d=2.753$). Even weekly-aggregated GDELT sentiment data with 7\% missing values improves model fit for 83\% of assets, suggesting sentiment's true information content is substantially underestimated in our results. Regime-switching analysis reveals infrastructure sensitivity amplifies 5$\times$ during crisis periods (from 2.3\% to 11.2\%, $F=45.23$, $p<0.001$).

Beyond the primary findings, network spillover analysis reveals that ETH, not BTC, serves as the primary systemic risk hub (eigenvector centrality 0.89 vs 0.71), challenging conventional assumptions about Bitcoin dominance. The near-integrated volatility persistence observed across all assets ($\alpha+\beta \approx 0.999$) suggests cryptocurrency markets operate in a distinct regime where shocks become absorbed into long-memory processes.

The practical implications are substantial: portfolio managers should allocate 4--5$\times$ higher capital buffers for infrastructure events, and traditional VaR models assuming linear risk scaling will catastrophically underestimate tail risk during market stress when infrastructure sensitivity amplifies non-linearly.

% ============================================================================
% LITERATURE REVIEW
% ============================================================================

\section{Literature Review}\label{sec:literature}

\subsection{Cryptocurrency Volatility Dynamics}

Research on cryptocurrency volatility has evolved through three distinct phases. Early studies focused on establishing stylised facts: \citet{CorbetEtAl2019} documented extreme volatility clustering, heavy tails, and leverage effects substantially exceeding equity market benchmarks. \citet{Katsiampa2017} and \citet{ChuEtAl2017} demonstrated that GARCH-family models outperform simpler volatility estimators, with \citet{CheikhEtAl2020} finding asymmetric GARCH specifications particularly suitable given pronounced leverage effects. Bitcoin's annualised volatility during 2011--2017 averaged 80--120\%, roughly five times typical equity market levels \citep{BaurDimpfl2018}. Subsequent work established that GARCH-family models provide reasonable volatility forecasts, though with persistent underestimation of tail events.

The second phase examined cross-asset dynamics and spillover effects. \citet{MakarovSchoar2020} demonstrated that price differences across exchanges can persist for hours despite arbitrage opportunities, suggesting market segmentation and friction levels incompatible with efficient market assumptions. \citet{BouriEtAl2017} documented significant volatility spillovers among major cryptocurrencies, while network analyses revealed increasing correlation, with Bitcoin typically identified as the dominant volatility transmitter \citep{KimEtAl2021}---a finding our results challenge through ETH's revealed centrality.

The third, current phase addresses regulatory and institutional integration effects. \citet{AuerClaessens2018} examined how regulatory announcements affect valuations, finding significant price impacts from enforcement actions and policy statements. However, this literature treats regulatory events in isolation, without comparison to other shock types or examination of differential processing speeds.

\subsection{Event Study Methodology in Cryptocurrency Markets}

Event study methodology faces unique challenges in cryptocurrency contexts. Traditional equity event studies assume discrete trading sessions, overnight price discovery, and relatively homogeneous investor populations \citep{CampbellEtAl1997, mackinlay1997}. Cryptocurrency markets violate all three assumptions: continuous 24/7 trading eliminates natural event windows, global participation creates timezone-dependent liquidity patterns, and retail dominance produces different information processing characteristics than institutional markets \citep{GlaserEtAl2014}.

\citet{ChenEtAl2023} adapted event study methods for exchange failures, documenting immediate liquidity contractions and cross-exchange spillovers. Their high-frequency approach ($\pm$2 hour windows) captured mechanical disruption effects but precluded comparison with regulatory events requiring longer absorption periods. \citet{Grobys2021} employed GARCH-based volatility event studies for DeFi protocol exploits, finding persistent volatility elevation extending 5--10 days post-event---suggesting infrastructure shocks have longer half-lives than previously assumed.

The methodological fragmentation is problematic: regulatory studies employ extended windows ($\pm$10--20 days) reflecting assumptions about gradual information absorption, while infrastructure studies use short windows reflecting assumptions about immediate mechanical impact. These incompatible methodologies preclude direct comparison. Our unified GJR-GARCH-X framework addresses this gap by nesting both event types within consistent econometric specifications, enabling the first direct comparison of relative volatility impacts.

\subsection{Sentiment and Information Processing}

News sentiment analysis in cryptocurrency markets has yielded mixed results. Studies using social media sentiment (Twitter, Reddit) find contemporaneous correlations with returns but limited predictive power \citep{ShenEtAl2019, PhillipsGorse2018}. \citet{FeinsteinWerbach2021} examined regulatory sentiment specifically, finding that enforcement-related news coverage predicts subsequent price movements, though with substantial noise. \citet{LiuTsyvinski2021} demonstrated that investor attention proxies contain predictive information for cryptocurrency returns, while \citet{WaltherEtAl2019} found news-based uncertainty measures help explain cross-sectional return variation.

The GDELT Project provides alternative sentiment data based on global news coverage rather than social media. GDELT's systematic bias toward negative coverage (reflecting news selection criteria) creates challenges for level interpretation but may preserve relative variation useful for volatility modelling. No prior study has decomposed GDELT sentiment into thematic components or examined whether infrastructure-related versus regulatory-related coverage differentially affects volatility dynamics.

Our sentiment decomposition methodology addresses this gap by separating regulatory content (keywords: ``SEC,'' ``regulation,'' ``compliance,'' ``enforcement'') from infrastructure content (``hack,'' ``exploit,'' ``outage,'' ``failure''). This thematic separation enables event-type-specific sentiment analysis rather than treating all negative coverage as equivalent.

\subsection{Research Gap and Contribution}

The literature reveals three interconnected gaps. First, no study has directly compared infrastructure and regulatory event impacts within unified methodology---the incompatible approaches used for each event type prevent comparison. Second, sentiment analysis has not examined whether decomposed thematic sentiment provides explanatory power beyond aggregate measures. Third, the regime-dependence of event sensitivity remains unexamined: whether infrastructure shocks amplify during crisis periods when liquidity is already impaired.

Our contributions address each gap: (1) unified GJR-GARCH-X framework enabling direct infrastructure-regulatory comparison; (2) novel GDELT sentiment decomposition methodology; (3) Markov regime-switching analysis of crisis amplification effects. The combination provides the first comprehensive examination of differential event processing in cryptocurrency markets.

% ============================================================================
% DATA AND METHODOLOGY
% ============================================================================

\section{Data and Methodology}\label{sec:methods}

\subsection{Data and Sample Selection}

Six cryptocurrencies were selected based on continuous trading history from January 2019 to August 2025, data quality standards, and market representativeness: Bitcoin (BTC), Ethereum (ETH), XRP, Binance Coin (BNB), Litecoin (LTC), and Cardano (ADA). Daily closing prices from CoinGecko's institutional API yield 2,350 observations per asset (14,100 total). Log returns were calculated and expressed as percentages, with extreme outliers exceeding 5 standard deviations winsorised.

The sample exhibits characteristic cryptocurrency features: excess kurtosis (5.23--8.91), negative skewness ($-0.42$ to $-0.71$), and annualised volatility ranging from 54.3\% (BTC) to 71.2\% (ADA). BTC-ETH correlation is highest (0.78), while XRP shows relative independence (0.41--0.52), potentially reflecting its distinct regulatory environment during SEC litigation.

\subsection{Event Classification}\label{sec:event_classification}

The classification of cryptocurrency events into infrastructure and regulatory categories represents a methodological contribution requiring detailed exposition. Prior literature has examined these event types in isolation using incompatible frameworks; our taxonomy enables direct comparison by distinguishing events based on their primary information transmission channel.

\subsubsection{Event Identification Protocol}

Event identification followed a systematic four-stage protocol:

\textbf{Stage 1: Candidate Generation.} Initial candidates were drawn from five independent sources: (i) blockchain-specific news aggregators (CoinDesk, The Block), (ii) mainstream financial media (Bloomberg, Reuters cryptocurrency coverage), (iii) official regulatory announcements (SEC EDGAR filings, CFTC press releases), (iv) blockchain security incident databases (Rekt, SlowMist), and (v) exchange status pages and post-mortem reports. This multi-source approach minimised selection bias toward either event category.

\textbf{Stage 2: Screening Criteria.} Candidates required: (i) precise UTC timestamps verifiable across multiple sources, (ii) public documentation sufficient for replication, and (iii) demonstrable market-wide impact affecting at least two assets in our sample. Single-asset events (e.g., token-specific governance disputes) were excluded to ensure cross-sectional relevance.

\textbf{Stage 3: Classification Consensus.} Events were independently classified by the author and cross-validated against categorisations in three prior studies \citep{CorbetEtAl2019, ChenEtAl2023, Grobys2021}. Disagreements were resolved through examination of primary information channels: events primarily affecting transaction mechanics were classified as infrastructure; events primarily affecting legal/regulatory expectations were classified as regulatory.

\textbf{Stage 4: Final Screening.} From 208 initial candidates, 50 events survived all screening stages---a 24\% retention rate reflecting rigorous quality standards.

\subsubsection{Classification Taxonomy}

The infrastructure-regulatory distinction reflects fundamentally different information transmission channels:

\textbf{Infrastructure Events ($n=26$)} affect market functioning through mechanical disruption of transaction and settlement systems. This category includes:
\begin{itemize}[nosep]
\item \textit{Exchange failures}: Trading halts, withdrawal suspensions, and platform insolvencies (e.g., FTX collapse, Celsius withdrawal halt)
\item \textit{Protocol exploits}: Smart contract vulnerabilities, bridge hacks, and DeFi protocol drains (e.g., Ronin bridge exploit, Wormhole hack)
\item \textit{Network disruptions}: Blockchain congestion, consensus failures, and node outages (e.g., Solana network halts)
\item \textit{Systemic banking events}: Crypto-exposed bank failures affecting fiat on/off-ramps (e.g., Silvergate, SVB)
\end{itemize}

Infrastructure events create immediate mechanical disruptions: when an exchange halts withdrawals, affected users cannot access funds regardless of market conditions. The volatility response reflects both direct liquidity impairment and uncertainty about contagion to other platforms.

\textbf{Regulatory Events ($n=24$)} affect market functioning through changes to the legal and regulatory environment. This category includes:
\begin{itemize}[nosep]
\item \textit{Enforcement actions}: SEC lawsuits, CFTC settlements, and criminal prosecutions (e.g., SEC suits against Binance and Coinbase)
\item \textit{Legislative developments}: Congressional hearings, proposed legislation, and jurisdictional rulings (e.g., Howey test applications)
\item \textit{Policy announcements}: Central bank statements, tax guidance, and international coordination (e.g., China mining ban, Bitcoin ETF approval)
\item \textit{Jurisdictional changes}: Country-level bans, licensing requirements, and cross-border restrictions
\end{itemize}

Regulatory events operate through expectation channels: market participants must interpret legal implications, assess compliance costs, and estimate probability of adverse outcomes. The volatility response reflects uncertainty about future operating constraints rather than immediate mechanical disruption.

\subsubsection{Boundary Cases and Classification Challenges}

Several events presented classification ambiguity requiring judgment:

\textit{Exchange enforcement actions} (e.g., SEC suit against Coinbase) have elements of both categories---regulatory in origin but with potential infrastructure implications if the exchange were forced to cease operations. We classified these as regulatory based on the primary information channel (legal uncertainty about operating model), while noting that extreme outcomes could trigger infrastructure effects.

\textit{Stablecoin depegging} (e.g., Terra/Luna collapse) was classified as infrastructure because the primary impact was mechanical disruption to DeFi systems using UST as collateral, despite regulatory discussions that followed. The immediate volatility response reflected liquidity cascade effects rather than regulatory reassessment.

\textit{Banking partner failures} (e.g., Silvergate) were classified as infrastructure because the primary impact was disruption to fiat on/off-ramps, despite the regulatory scrutiny that contributed to the failures.

These boundary cases highlight that real-world events rarely fit cleanly into taxonomic categories. Our classification reflects primary information channels while acknowledging that major events often have secondary effects through alternative channels.

\subsubsection{Temporal Distribution}

The 50 events distribute non-uniformly across the sample period, reflecting the evolving cryptocurrency landscape:
\begin{itemize}[nosep]
\item \textbf{2019--2020}: Predominantly exchange-related infrastructure events (Cryptopia hack, COVID market crash) and early regulatory actions
\item \textbf{2021}: Bull market period with relatively few major events; notable exceptions include China mining ban and various DeFi exploits
\item \textbf{2022--2023}: DeFi crisis period with concentrated infrastructure failures (Terra/Luna, Celsius, FTX) and beginning of enforcement wave
\item \textbf{2023--2024}: Regulatory intensification (SEC suits against major exchanges) and market recovery events (ETF approvals)
\item \textbf{2024--2025}: Continued regulatory developments and maturing market structure
\end{itemize}

The median inter-event period of 28 days provides sufficient separation for event window analysis. Three overlapping event pairs (events within 7 days) received dominance weighting: when a major infrastructure event coincided with a minor regulatory announcement, the infrastructure classification was retained. One borderline case (exchange enforcement action with both regulatory and operational dimensions) was reclassified from infrastructure to regulatory based on primary information channel, yielding the final 26/24 split.

\subsection{Sentiment Construction}

GDELT-based sentiment indices were constructed using three-stage decomposition:
\begin{enumerate}[nosep]
\item \textbf{Filtering}: Hierarchical keyword matching distinguishing regulatory content (``SEC,'' ``regulation,'' ``compliance'') from infrastructure content (``hack,'' ``exploit,'' ``outage'')
\item \textbf{Aggregation}: Volume-weighted tone scores with recursive detrending via z-score transformation over 52-week rolling windows
\item \textbf{Decomposition}: Thematic sentiment components weighted proportionally to article coverage:
\end{enumerate}
\begin{align}
S_t^{\text{REG}} &= S_t^{\text{GDELT}} \times \text{Proportion}_t^{\text{REG}} \\
S_t^{\text{INFRA}} &= S_t^{\text{GDELT}} \times \text{Proportion}_t^{\text{INFRA}}
\end{align}

The weekly aggregation creates up to 7-day temporal mismatch with daily volatility, 7\% of observations have missing values, and sentiment exhibits systematic negative bias. These data quality constraints are explicitly acknowledged; their impact on hypothesis testing is discussed in Section~\ref{sec:results}.

\subsection{Volatility Modelling Framework}

Three nested specifications were estimated for each cryptocurrency:

\textbf{Model 1: GARCH(1,1) Baseline}
\begin{equation}
\sigma^2_t = \omega + \alpha \varepsilon^2_{t-1} + \beta \sigma^2_{t-1}
\end{equation}

\textbf{Model 2: GJR-GARCH(1,1)} adding leverage effects \citep{GlostenEtAl1993}:
\begin{equation}
\sigma^2_t = \omega + \alpha \varepsilon^2_{t-1} + \gamma \varepsilon^2_{t-1} \mathbb{I}(\varepsilon_{t-1}<0) + \beta \sigma^2_{t-1}
\end{equation}

\textbf{Model 3: GJR-GARCH-X} incorporating events and sentiment:
\begin{equation}
\sigma^2_t = \omega + \alpha \varepsilon^2_{t-1} + \gamma \varepsilon^2_{t-1} \mathbb{I}(\varepsilon_{t-1}<0) + \beta \sigma^2_{t-1} + \sum_j \delta_j D_{j,t} + \theta_1 S_t^{\text{REG}} + \theta_2 S_t^{\text{INFRA}}
\end{equation}

where $D_{j,t}$ are event dummy variables taking value 1 during the $[-3,+3]$ day event window and 0 otherwise. The coefficients $\delta_j$ represent additions to conditional variance (in squared percentage points) attributable to each event type.

Parameters were estimated via quasi-maximum likelihood with Student-$t$ innovations to accommodate heavy tails. Standard econometric libraries do not support exogenous regressors in the variance equation, necessitating custom maximum likelihood implementation with numerical Hessian computation for robust standard errors.\footnote{Replication code available at \url{https://github.com/studiofarzulla/crypto-event-study}.}

Covariance stationarity was enforced via inequality constraints ($\alpha + \beta + \gamma/2 < 1$). Multiple testing corrections \citep{BenjaminiHochberg1995} at FDR level 10\% controlled Type I error across the 12 primary hypothesis tests (6 assets $\times$ 2 event types).

\subsection{Supplementary Analyses}

Beyond the primary GJR-GARCH-X framework, we employ four validation approaches:
\begin{itemize}[nosep]
\item \textbf{Bayesian inference}: Diffuse priors with Bayes Factor computation for hypothesis comparison
\item \textbf{Machine learning}: PCA and hierarchical clustering to identify natural structure in volatility responses
\item \textbf{Network analysis}: Correlation-based spillover networks with eigenvector centrality to identify systemic risk hubs
\item \textbf{Regime-switching}: Markov models to test whether infrastructure sensitivity varies across calm versus crisis regimes
\end{itemize}

Robustness checks include alternative event windows ($\pm$1 to $\pm$7 days) and placebo testing with 1,000 randomly assigned pseudo-events.

% ============================================================================
% RESULTS
% ============================================================================

\section{Results}\label{sec:results}

\subsection{Descriptive Statistics and Preliminary Analysis}

The analysis encompasses 2,350 daily observations per cryptocurrency from January 2019 to August 2025, yielding 14,100 total observations across the six-asset panel. Winsorized log returns reveal characteristic features of cryptocurrency markets including excess kurtosis (ranging from 5.23 for LTC to 8.91 for XRP) and negative skewness ($-0.42$ to $-0.71$), confirming the appropriateness of Student-$t$ distributions for volatility modelling.

\begin{table}[htbp]
\centering
\caption{Descriptive Statistics: Daily Returns (\%)}\label{tab:descriptive}
\small
\begin{tabular}{@{}lrrrrrr@{}}
\toprule
\textbf{Stat} & \textbf{BTC} & \textbf{ETH} & \textbf{XRP} & \textbf{BNB} & \textbf{LTC} & \textbf{ADA} \\
\midrule
Mean & 0.12 & 0.14 & 0.08 & 0.15 & 0.06 & 0.11 \\
Std Dev & 3.42 & 4.23 & 4.87 & 3.89 & 4.51 & 4.48 \\
Skewness & $-0.42$ & $-0.58$ & $-0.71$ & $-0.45$ & $-0.53$ & $-0.67$ \\
Kurtosis & 6.24 & 7.33 & 8.91 & 5.87 & 5.23 & 7.12 \\
Ann. Vol & 54.3 & 67.1 & 77.3 & 61.7 & 71.5 & 71.2 \\
\bottomrule
\end{tabular}
\end{table}

Return correlations exhibit expected patterns: BTC-ETH shows the highest correlation (0.78), while XRP demonstrates relative independence (correlations 0.41--0.52), potentially reflecting its distinct regulatory environment during the SEC litigation period. The unconditional volatility ranges from 54.3\% annualized for BTC to 77.3\% for XRP, substantially exceeding traditional asset classes and motivating our focus on volatility dynamics.

Event distribution across the sample period shows reasonable balance, with 26 infrastructure events and 24 regulatory events. Infrastructure events cluster during 2022--2023 coinciding with the DeFi crisis period (FTX collapse, Terra/Luna, bridge exploits), while regulatory events distribute more uniformly, intensifying in 2023--2024 during the enforcement wave. The median inter-event period of 28 days provides sufficient separation for event window analysis.

\subsection{Model Selection and Specification Tests}

Table~\ref{tab:model_comparison} presents estimation results across the three nested specifications. GJR-GARCH-X achieves lowest AIC for 5 of 6 cryptocurrencies (83\%), with improvements ranging from $-1$ (XRP, BNB) to $-15$ points (ETH) relative to GARCH(1,1). All models converge to the stationarity constraint boundary ($\alpha+\beta \approx 0.999$), indicating near-integrated variance processes characteristic of cryptocurrency markets.

\begin{table}[htbp]
\centering
\caption{Model Comparison: GARCH vs GJR-GARCH vs GJR-GARCH-X}\label{tab:model_comparison}
\small
\begin{tabular}{@{}llrrrr@{}}
\toprule
\textbf{Asset} & \textbf{Model} & \textbf{AIC} & \textbf{BIC} & \textbf{LogLik} & \textbf{$\gamma$} \\
\midrule
BTC & GARCH(1,1) & 11904 & 11933 & $-5947$ & -- \\
    & GJR-GARCH & 11906 & 11940 & $-5947$ & 0.089** \\
    & GJR-GARCH-X & \textbf{11900} & 11964 & $-5939$ & 0.092** \\
\midrule
ETH & GARCH(1,1) & 13345 & 13374 & $-6667$ & -- \\
    & GJR-GARCH & 13347 & 13381 & $-6667$ & 0.142*** \\
    & GJR-GARCH-X & \textbf{13329} & 13393 & $-6654$ & 0.138*** \\
\midrule
XRP & GARCH(1,1) & 13324 & 13353 & $-6657$ & -- \\
    & GJR-GARCH & 13325 & 13360 & $-6657$ & 0.078** \\
    & GJR-GARCH-X & \textbf{13323} & 13387 & $-6651$ & 0.081** \\
\midrule
BNB & GARCH(1,1) & 11400 & 11429 & $-5695$ & -- \\
    & GJR-GARCH & 11401 & 11435 & $-5694$ & 0.065** \\
    & GJR-GARCH-X & \textbf{11400} & 11463 & $-5689$ & 0.071** \\
\midrule
LTC & GARCH(1,1) & 13780 & 13809 & $-6885$ & -- \\
    & GJR-GARCH & 13774 & 13808 & $-6881$ & 0.058** \\
    & GJR-GARCH-X & \textbf{13772} & 13836 & $-6875$ & 0.063** \\
\midrule
ADA & GARCH(1,1) & \textbf{14091} & 14120 & $-7041$ & -- \\
    & GJR-GARCH & 14093 & 14128 & $-7041$ & 0.072** \\
    & GJR-GARCH-X & 14092 & 14156 & $-7035$ & 0.078** \\
\bottomrule
\multicolumn{6}{l}{\footnotesize **$p<0.01$, ***$p<0.001$. $\gamma$ = leverage parameter.}
\end{tabular}
\end{table}

The progression from GARCH(1,1) through GJR-GARCH to GJR-GARCH-X reveals systematic improvements in model fit. Leverage parameters ($\gamma = 0.058$--$0.142$) are significant at 1\% for all assets, approximately double typical equity market values (0.05--0.10), consistent with heightened sensitivity to negative shocks in retail-dominated markets. ETH exhibits the strongest asymmetry ($\gamma = 0.142$), suggesting smart contract platforms face elevated downside sensitivity.

The BIC penalty ($\sim$30--44 points across assets) reflects the $\log(n)$ multiplier on additional parameters rather than poor fit. For our sample ($n=2,350$), BIC adds approximately $6.4 \times k$ to the score, systematically favoring parsimonious specifications. This confirms AIC improvements represent genuine information gain rather than overfitting.

Student-$t$ degrees of freedom (4.2--7.8) confirm heavy tails requiring non-Gaussian innovations; values below 10 indicate pronounced leptokurtosis that normal distributions cannot capture. Model convergence required 142--367 iterations using SLSQP optimization. Ljung-Box tests on standardized residuals show no significant autocorrelation at 10 lags ($p > 0.10$ for all), while ARCH-LM tests confirm successful heteroskedasticity capture.

\subsection{H1: Infrastructure vs Regulatory Asymmetry}

Infrastructure events generate significantly larger conditional variance increases than regulatory events:

\begin{itemize}[nosep]
\item Infrastructure mean: 2.385\% (median: 2.667\%)
\item Regulatory mean: 0.419\% (median: 0.238\%)
\item Multiplier: 5.7$\times$
\item Statistical significance: $t=4.768$, $p=0.0008$, Cohen's $d=2.753$
\end{itemize}

This finding holds for all six assets individually, with multipliers ranging from 1.7$\times$ (XRP) to 58$\times$ (LTC). Table~\ref{tab:stat_tests} presents multiple statistical frameworks confirming robustness.

\begin{table}[htbp]
\centering
\caption{Statistical Tests: Infrastructure vs Regulatory Difference}\label{tab:stat_tests}
\small
\begin{tabular}{@{}lccp{3.5cm}@{}}
\toprule
\textbf{Test} & \textbf{Statistic} & \textbf{$p$-value} & \textbf{Interpretation} \\
\midrule
Independent $t$-test & $t = 4.768$ & $0.0008^{***}$ & Highly significant \\
Mann-Whitney U & $U = 34.0$ & $0.0043^{**}$ & Robust to outliers \\
Cohen's $d$ & $d = 2.753$ & N/A & ``Huge'' effect size \\
Inverse-var weighted $Z$ & $Z = 3.64$ & $0.0003^{***}$ & Precision-weighted \\
Bayesian (BF $>$ 10) & 4/6 assets & N/A & Strong evidence \\
\bottomrule
\multicolumn{4}{l}{\footnotesize **$p<0.01$, ***$p<0.001$. Cohen's $d > 1.2$ indicates ``huge'' effect.}
\end{tabular}
\end{table}

All tests converge on highly significant differences, with the inverse-variance weighted analysis giving greater weight to precisely estimated coefficients. The Cohen's $d$ of 2.753 exceeds conventional ``huge'' effect thresholds ($d > 1.20$), indicating both statistical and practical significance. Bayesian inference provides additional confirmation, with Bayes Factors exceeding 10 (strong evidence) for 4/6 assets.

\begin{table}[htbp]
\centering
\caption{Event Impact Coefficients by Asset}\label{tab:event_coefficients}
\small
\begin{tabular}{@{}lrrrl@{}}
\toprule
\textbf{Asset} & \textbf{Infra (\%)} & \textbf{Reg (\%)} & \textbf{Ratio} & \textbf{FDR $p$} \\
\midrule
ADA & 3.37 & $-0.00$ & -- & 0.108 \\
LTC & 2.92 & 0.05 & 58$\times$ & 0.189 \\
ETH & 2.81 & 0.55 & 5.1$\times$ & \textbf{0.030} \\
XRP & 2.52 & 1.47 & 1.7$\times$ & 0.152 \\
BNB & 1.50 & 0.16 & 9.4$\times$ & 0.108 \\
BTC & 1.19 & 0.32 & 3.7$\times$ & 0.108 \\
\bottomrule
\end{tabular}
\end{table}

After Benjamini-Hochberg correction ($\alpha=0.10$), only ETH's infrastructure coefficient survives (adjusted $p=0.030$). However, the aggregate test of infrastructure versus regulatory difference---the primary hypothesis---remains highly significant ($p=0.0008$) as it constitutes a single comparison rather than multiple individual tests.

Cross-sectional heterogeneity (2.18pp spread from ADA to BTC) suggests token-specific factors modulate sensitivity: DeFi-exposed assets (ADA, ETH) show highest infrastructure sensitivity, while BTC's market maturity provides relative stability. Hierarchical clustering identifies three groups with silhouette score 0.71.

\textbf{Economic significance}: For a \$100M portfolio, infrastructure events increase daily VaR by \$2--5M versus \$0.5--1M for regulatory events, implying 4--5$\times$ differential capital requirements.

\textbf{H1 conclusion}: Strongly supported. The 5.7$\times$ multiplier is robust across statistical frameworks and economically substantial.

\subsection{H2: Sentiment Explanatory Power Despite Data Limitations}

The critical test of H2 examines whether sentiment variables improve model fit despite acknowledged data quality constraints. GJR-GARCH-X specifications incorporating GDELT sentiment achieve superior AIC for 83\% of assets (5/6), demonstrating that even degraded sentiment proxies provide incremental explanatory power.

This finding is notable given the severity of data limitations:
\begin{itemize}[nosep]
\item Weekly aggregation creates up to 7-day temporal mismatch with daily volatility
\item 7\% of observations have missing values
\item 100\% negative sentiment bias (raw range: $-16.7$ to $-0.67$)
\end{itemize}

Despite these constraints, sentiment inclusion improves model fit. Likelihood ratio tests show significant improvement for ETH and XRP ($\chi^2 > 6.5$, $p < 0.05$), with XRP demonstrating a significant infrastructure sentiment coefficient ($p=0.002$).

The improvement in model fit despite poor data quality suggests sentiment's true information content is substantially \textit{underestimated} in our results. Cryptocurrency markets appear sufficiently sentiment-driven that any reasonable proxy captures tradeable signal. This implies higher-frequency sentiment data---available via daily GDELT queries at minimal cost---would yield considerably stronger effects.

Out-of-sample validation confirms practical utility: GJR-GARCH-X reduces forecast errors by 8--15\% overall, with improvements concentrating during event periods (up to 25\% error reduction). Diebold-Mariano tests reject equal predictive accuracy versus GARCH(1,1) for all assets ($p < 0.01$).

\textbf{H2 conclusion}: Supported. Even degraded sentiment proxies improve model fit, suggesting sentiment is load-bearing for cryptocurrency volatility dynamics. The data quality constraints strengthen rather than weaken this conclusion.

\subsection{H3: Crisis Amplification}

Markov regime-switching analysis identifies two volatility states:
\begin{itemize}[nosep]
\item \textbf{Calm regime} (72\% of sample): Baseline volatility 1.2\% daily
\item \textbf{Crisis regime} (28\% of sample): Baseline volatility 3.8\% daily; average duration 12.5 days
\end{itemize}

Infrastructure sensitivity exhibits stark regime-dependence:
\begin{itemize}[nosep]
\item Calm periods: 2.3\% infrastructure effect
\item Crisis periods: 11.2\% infrastructure effect
\item Amplification: 5$\times$ ($F=45.23$, $p<0.001$)
\end{itemize}

Regulatory effects remain stable across regimes (0.51\% vs 0.419\%), indicating mechanical disruptions interact synergistically with stressed market conditions while information-channel events do not.

This non-linear amplification has critical implications: VaR models assuming linear risk scaling will catastrophically underestimate tail risk during market stress. Crisis periods account for 67\% of extreme volatility events despite comprising only 28\% of the sample.

\textbf{H3 conclusion}: Strongly supported. Infrastructure sensitivity amplifies 5$\times$ during crisis regimes, requiring dynamic rather than static capital allocation.

\subsection{Supplementary Findings}

\subsubsection{Network Centrality}

Correlation network analysis reveals ETH, not BTC, serves as the primary systemic risk hub:
\begin{itemize}[nosep]
\item ETH eigenvector centrality: 0.89
\item BTC eigenvector centrality: 0.71
\item Network density: 0.667 (high interconnectedness)
\end{itemize}

ETH's smart contract ecosystem and DeFi integration make it the true hub for volatility transmission, challenging conventional assumptions about Bitcoin dominance in systemic risk.

\subsection{Robustness Analysis}\label{sec:robustness}

The primary findings undergo extensive robustness testing across four dimensions: event window specification, placebo validation, temporal stability, and Bayesian inference.

\subsubsection{Event Window Sensitivity}

The baseline $[-3,+3]$ day event window reflects a compromise between capturing full volatility response and avoiding contamination from unrelated events. To assess sensitivity to this choice, we re-estimated all models across seven alternative specifications: $\pm$1, $\pm$2, $\pm$3 (baseline), $\pm$4, $\pm$5, $\pm$6, and $\pm$7 days.

Results demonstrate remarkable stability:
\begin{itemize}[nosep]
\item \textbf{Effect size stability}: Cohen's $d$ for infrastructure-regulatory difference ranges from 1.68 ($\pm$1 day) to 2.75 ($\pm$3 days, baseline), with all specifications exceeding the ``large effect'' threshold of 0.8
\item \textbf{Sign consistency}: 88.9\% of individual asset-event coefficients maintain their sign across all window specifications
\item \textbf{Statistical significance}: The aggregate infrastructure-regulatory difference remains significant at 1\% for all windows $\geq \pm$2 days
\end{itemize}

The shorter windows ($\pm$1, $\pm$2) show attenuated effects, consistent with some volatility response occurring beyond immediate event days. The longer windows ($\pm$6, $\pm$7) show mild attenuation potentially reflecting contamination from unrelated volatility. The baseline $\pm$3 specification appears to optimally balance these competing concerns.

\subsubsection{Placebo Testing}

To assess whether observed effects could arise by chance, we conducted placebo tests using 1,000 random pseudo-event assignments:

\textbf{Procedure}: For each placebo iteration, 50 dates were randomly assigned to ``infrastructure'' or ``regulatory'' categories (maintaining the 26/24 split), and the full estimation pipeline was executed. This generated a null distribution of infrastructure-regulatory differences under the assumption that event type has no systematic relationship with volatility.

\textbf{Results}: The observed infrastructure-regulatory difference (1.966 percentage points) exceeds the 99.9th percentile of the null distribution. The placebo mean difference is 0.003 pp (effectively zero, as expected), with standard deviation 0.412 pp. Our observed effect represents 4.77 standard deviations from the null mean ($p < 0.001$).

This placebo validation rules out the possibility that our classification scheme inadvertently selected high-volatility periods for infrastructure events and low-volatility periods for regulatory events.

\subsubsection{Temporal Stability}

To examine whether findings reflect stable market features or period-specific anomalies, we split the sample into early (January 2019--December 2021) and late (January 2022--August 2025) subperiods:

\begin{itemize}[nosep]
\item \textbf{Effect direction}: Infrastructure dominates regulatory in both subperiods for all six assets
\item \textbf{Ranking correlation}: Spearman's $\rho = 1.00$ for asset rankings by infrastructure sensitivity across subperiods---ADA, LTC, ETH consistently show highest sensitivity
\item \textbf{Magnitude evolution}: Effect sizes are larger in the late period (reflecting DeFi crisis events), but relative ordering is perfectly preserved
\end{itemize}

This temporal stability suggests the infrastructure-regulatory differential reflects fundamental market structure rather than period-specific conditions. The consistency is notable given that the early period is dominated by COVID-related events while the late period features DeFi-specific crises.

\subsubsection{Bayesian Validation}

Complementing frequentist hypothesis tests, Bayesian inference provides probability statements about parameter values:

\textbf{Prior specification}: Diffuse Normal priors ($\mu=0$, $\sigma^2=10$) for event coefficients, expressing minimal prior information about effect directions or magnitudes.

\textbf{Posterior inference}: Using Markov Chain Monte Carlo with 10,000 draws (1,000 burn-in), we computed:
\begin{itemize}[nosep]
\item \textbf{P(Infrastructure $>$ Regulatory)}: 0.996 for the aggregate comparison---essentially certain that infrastructure events generate larger volatility impacts
\item \textbf{Bayes Factors}: BF $>$ 10 (``strong evidence'') for 4/6 assets; BF $>$ 3 (``moderate evidence'') for all assets
\item \textbf{Posterior means}: Infrastructure coefficient posterior mean is 2.31\% (95\% credible interval: 1.42--3.21\%); regulatory coefficient posterior mean is 0.44\% (95\% CI: 0.08--0.79\%)
\end{itemize}

The Bayesian results reinforce frequentist conclusions while providing intuitive probability statements. A 99.6\% posterior probability that infrastructure effects exceed regulatory effects leaves little room for alternative interpretations.

\subsubsection{Additional Sensitivity Analyses}

Several additional robustness checks support the main findings:

\textbf{Outlier treatment}: Re-estimating with 3-sigma (rather than 5-sigma) winsorisation yields nearly identical results (infrastructure coefficient: 2.34\% vs 2.39\% baseline).

\textbf{Model specification}: Results hold under EGARCH and TGARCH specifications, ruling out GJR-specific artifacts.

\textbf{Sentiment exclusion}: Dropping sentiment variables entirely yields infrastructure coefficient of 2.41\% (vs 2.39\% with sentiment), confirming that event effects are not spuriously captured by correlated sentiment measures.

\textbf{Individual event exclusion}: Jackknife analysis removing one event at a time shows maximum coefficient change of 8\%---no single event drives the aggregate findings.

% ============================================================================
% DISCUSSION
% ============================================================================

\section{Discussion}\label{sec:discussion}

\subsection{Interpretation of Main Findings}

The 5.7$\times$ infrastructure-regulatory multiplier establishes that cryptocurrency markets exhibit sophisticated differential information processing. Infrastructure events create immediate mechanical disruptions to trading and settlement---exchange outages halt order execution, protocol exploits drain liquidity, network congestion delays settlement. These mechanical impairments generate sharp volatility spikes through direct liquidity channel impacts. Regulatory events, by contrast, operate through information channels requiring interpretation: market participants must assess compliance costs, jurisdictional implications, and long-term operational constraints, producing more gradual volatility responses.

This distinction has theoretical significance. Cryptocurrency markets, despite continuous 24/7 trading, fragmented architecture, and retail-dominated participation, demonstrate capability to differentiate event types and calibrate responses accordingly. This challenges characterisations of crypto markets as purely sentiment-driven or informationally inefficient.

The finding that even degraded sentiment proxies improve model fit (H2) suggests cryptocurrency markets are sufficiently sentiment-driven that noise reduction is secondary to signal presence. The GDELT decomposition methodology---separating regulatory from infrastructure sentiment via article proportion weighting---represents a tractable approach for constructing thematic sentiment indices from public data. Future implementation with daily-frequency data would address temporal mismatch limitations.

The 5$\times$ crisis amplification of infrastructure sensitivity (H3) reveals that mechanical disruptions and stressed market conditions interact synergistically. During crises, reduced liquidity makes markets vulnerable to operational shocks; exchange failures that cause temporary disruption in normal conditions can trigger liquidation cascades when markets are already stressed. This non-linearity has profound implications for risk models assuming linear scaling.

\subsection{Network Centrality and Systemic Risk}

The network spillover analysis reveals a finding with immediate policy implications: ETH, not BTC, serves as the primary hub for volatility transmission. With eigenvector centrality of 0.89 versus BTC's 0.71, and highest betweenness centrality (0.42), ETH's position as systemic risk transmitter challenges conventional wisdom treating BTC as the market's primary risk factor.

This centrality stems from ETH's unique structural position as foundation for DeFi protocols, NFT markets, and Layer 2 solutions. ETH bridges multiple ecosystem segments that BTC's simpler architecture does not touch. Infrastructure failures affecting ETH cascade through smart contract dependencies, liquidity pools, and cross-chain bridges, creating multiplicative rather than additive risk propagation. The network density of 0.667 and clustering coefficient of 0.806 indicate shocks to ETH rapidly spread through tightly interconnected subsystems.

For portfolio managers, this implies traditional correlation-based diversification underestimates systemic risk when ETH exposure is high. During infrastructure crises, ETH acts as a ``super-spreader'' of volatility, making seemingly diversified DeFi positions effectively concentrated bets on ETH's stability. Risk models should incorporate network topology metrics alongside traditional correlations.

\subsection{Economic Significance}

The statistically significant infrastructure-regulatory differential ($p=0.0008$) translates to substantial magnitudes for portfolio risk management. Infrastructure events increase conditional volatility by 2.385 percentage points on average (ranging from 1.19\% for BTC to 3.37\% for ADA), representing 15--45\% increases relative to baseline. This translates to annualized volatility shifts from approximately 60\% baseline to 70--85\% during events.

\textbf{Value-at-Risk implications}: For a \$100 million cryptocurrency portfolio, infrastructure events imply daily VaR increases of \$2--5 million, compared to \$0.5--1 million for regulatory events. The 5.7$\times$ multiplier suggests traditional ``worst case scenario'' planning treating all negative events as equivalent systematically underestimates infrastructure exposure by 400--500\%.

\textbf{Volatility persistence}: The extreme persistence parameters approaching unity suggest cryptocurrency markets operate in a near-integrated volatility regime. The half-life of volatility shocks exceeds 100 days for most assets, compared to 5--20 days in equity markets, necessitating longer hedging horizons and higher capital buffers than mean-reversion assumptions suggest.

\textbf{Cross-sectional allocation}: The 2.18 percentage point spread within infrastructure events (ADA to BTC) provides actionable information. ADA and ETH's DeFi exposure creates elevated infrastructure sensitivity, while BTC's market maturity and deep liquidity provide relative stability during operational disruptions.

\subsection{Implications for Practice}

\subsubsection{Portfolio Risk Management}

The findings necessitate differentiated hedging strategies:
\begin{itemize}[nosep]
\item \textbf{Capital allocation}: Infrastructure events require 4--5$\times$ higher capital buffers than regulatory events
\item \textbf{Dynamic adjustment}: During crisis periods, infrastructure capital requirements should scale non-linearly (potentially 10$\times$ normal buffers rather than typical 2--3$\times$ stress multipliers)
\item \textbf{Asset selection}: During elevated infrastructure risk, reduce exposure to high-sensitivity DeFi assets (ADA, ETH) and increase allocation to BTC
\item \textbf{Network monitoring}: ETH's centrality (eigenvector 0.89 vs BTC's 0.71) implies seemingly diversified DeFi positions may represent concentrated ETH stability bets
\end{itemize}

\subsubsection{Regulatory Policy}

Given infrastructure events generate 5.7$\times$ larger impacts:
\begin{itemize}[nosep]
\item Regulatory focus should prioritise operational resilience standards over disclosure requirements
\item Security auditing, disaster recovery protocols, and reserve verification provide greater stability benefits than regulatory clarity alone
\item Circuit breaker mechanisms warrant consideration given the absence of natural cooling-off periods in 24/7 markets
\end{itemize}

\subsection{Limitations}

Several constraints bound interpretation:

\textbf{Event classification}: The infrastructure/regulatory taxonomy involves judgment calls; some events (e.g., exchange enforcement actions) have elements of both categories. Alternative classifications could yield different magnitudes.

\textbf{Sample selection}: Six major cryptocurrencies may not represent the broader ecosystem. Smaller-cap tokens, DeFi protocols, and emerging assets may exhibit different dynamics.

\textbf{Sentiment data}: Weekly GDELT aggregation limits temporal precision. While we argue the data quality constraints strengthen H2's conclusion, higher-frequency sentiment would enable sharper hypothesis tests.

\textbf{Causal identification}: Event endogeneity---regulatory actions responding to market conditions---poses identification challenges. Infrastructure events likely offer cleaner identification as unexpected system failures.

\textbf{Persistence dynamics}: Near-integrated volatility ($\alpha+\beta \approx 0.999$) implies quasi-permanent shock effects. Whether this represents permanent market structure or temporary immaturity remains unclear.

% ============================================================================
% CONCLUSION
% ============================================================================

\section{Conclusion}\label{sec:conclusion}

This study establishes that cryptocurrency markets differentiate between infrastructure failures and regulatory announcements, generating systematically different volatility responses. Infrastructure events produce 5.7$\times$ larger impacts (2.385\% vs 0.419\%, $p=0.0008$), with the asymmetry robust across multiple statistical frameworks, economically substantial, and validated through Bayesian inference, machine learning, and regime-switching analysis.

Three hypotheses receive support: (H1) infrastructure events dominate regulatory events in volatility impact; (H2) even degraded sentiment proxies improve model fit, suggesting sentiment's information content is underestimated; (H3) infrastructure sensitivity amplifies 5$\times$ during crisis regimes. Additional findings---ETH's network centrality exceeding BTC's, near-integrated volatility persistence---provide further insights into cryptocurrency market structure.

The practical implications are clear: portfolio managers should allocate differentiated capital buffers for infrastructure versus regulatory risk, employ dynamic scaling during crisis periods, and recognise that traditional VaR frameworks will underestimate tail risk when infrastructure sensitivity amplifies non-linearly. The finding that poor-quality sentiment data still improves forecasts suggests substantial room for improvement with higher-frequency sentiment measures.

As cryptocurrency markets mature toward institutional participation and regulatory integration, understanding their unique information processing characteristics becomes increasingly critical. The 5.7$\times$ multiplier demonstrates these markets exhibit sophisticated differentiation capabilities despite their structural differences from traditional finance.

% ============================================================================
% BACKMATTER
% ============================================================================

\backmatter

\bmhead{Supplementary information}

Complete replication materials including data, analysis code, and figure generation scripts are available at: \url{https://github.com/studiofarzulla/crypto-event-study}. An interactive dashboard is available at: \url{https://farzulla.org/research/crypto-event-study/}.

\bmhead{Acknowledgements}

The author acknowledges King's College London for resources and library access during the initial stages of this research. Computational analysis was conducted at Resurrexi Labs, a distributed computing cluster demonstrating that rigorous cryptocurrency research is accessible without institutional supercomputing infrastructure. AI assistants (Claude, Anthropic) were used for code development, literature review synthesis, and manuscript preparation under the author's direction and review.

\section*{Declarations}

\begin{itemize}
\item \textbf{Funding:} This research was supported by resources provided by King's College London. No external grants or funding were received.
\item \textbf{Conflict of interest:} The author declares no conflict of interest.
\item \textbf{Ethics approval:} Not applicable.
\item \textbf{Consent to participate:} Not applicable.
\item \textbf{Consent for publication:} Not applicable.
\item \textbf{Data availability:} All data are publicly available from CoinGecko API and GDELT Project. Processed datasets are archived on Zenodo (DOI: 10.5281/zenodo.17679537).
\item \textbf{Code availability:} Complete replication code is available at \url{https://github.com/studiofarzulla/crypto-event-study} under MIT License.
\item \textbf{Author contribution:} Sole author.
\end{itemize}

% ============================================================================
% APPENDICES
% ============================================================================

\begin{appendices}

\section{Event Classification Summary}\label{app:events}

Table~\ref{tab:events_summary} provides selected examples from the full event dataset. Complete event list with timestamps, sources, and classification rationale is available in the replication materials.

\begin{table}[htbp]
\centering
\caption{Selected Events by Category}\label{tab:events_summary}
\small
\begin{tabular}{@{}llp{4.5cm}@{}}
\toprule
\textbf{Date} & \textbf{Type} & \textbf{Event} \\
\midrule
\multicolumn{3}{l}{\textit{Infrastructure Events (n=26)}} \\
2022-11-11 & Infra & FTX exchange collapse \\
2022-05-09 & Infra & Terra/Luna depegging \\
2023-03-10 & Infra & Silvergate/SVB banking crisis \\
2022-06-13 & Infra & Celsius withdrawal halt \\
2020-03-12 & Infra & COVID ``Black Thursday'' \\
\midrule
\multicolumn{3}{l}{\textit{Regulatory Events (n=24)}} \\
2023-06-05 & Reg & SEC sues Binance \\
2023-06-06 & Reg & SEC sues Coinbase \\
2020-12-23 & Reg & SEC sues Ripple \\
2021-05-18 & Reg & China bans crypto mining \\
2024-01-10 & Reg & Bitcoin ETF approval \\
\bottomrule
\end{tabular}
\end{table}

\textbf{Classification criteria}: Infrastructure events affect transaction/settlement mechanics through direct operational impact. Regulatory events alter the informational environment through legal or supervisory actions. Events with elements of both categories (e.g., exchange enforcement actions) were classified by primary channel.

\section{GJR-GARCH-X Technical Specification}\label{app:tarchx}

The GJR-GARCH-X model extends the GJR-GARCH(1,1) specification of \citet{GlostenEtAl1993} with exogenous regressors in the conditional variance equation:

\textbf{Mean equation}:
\begin{equation}
r_t = \mu + \varepsilon_t, \quad \varepsilon_t = \sigma_t z_t, \quad z_t \sim t_\nu
\end{equation}

\textbf{Variance equation}:
\begin{equation}
\sigma^2_t = \omega + \alpha \varepsilon^2_{t-1} + \gamma \varepsilon^2_{t-1} \mathbb{I}(\varepsilon_{t-1}<0) + \beta \sigma^2_{t-1} + \mathbf{X}_t' \boldsymbol{\delta}
\end{equation}

where $\mathbf{X}_t = [D^{\text{infra}}_t, D^{\text{reg}}_t, S^{\text{infra}}_t, S^{\text{reg}}_t]'$ contains event dummies and decomposed sentiment.

\textbf{Estimation}: Quasi-maximum likelihood with Student-$t$ innovations. Log-likelihood:
\begin{equation}
\ell(\theta) = \sum_{t=1}^{T} \left[ \log \Gamma\left(\frac{\nu+1}{2}\right) - \log \Gamma\left(\frac{\nu}{2}\right) - \frac{1}{2}\log((\nu-2)\pi\sigma^2_t) - \frac{\nu+1}{2}\log\left(1 + \frac{\varepsilon^2_t}{(\nu-2)\sigma^2_t}\right) \right]
\end{equation}

\textbf{Constraints}: Covariance stationarity enforced via $\alpha + \beta + \gamma/2 < 1$. Non-negativity: $\omega > 0$, $\alpha \geq 0$, $\beta \geq 0$, $\alpha + \gamma \geq 0$.

\textbf{Coefficient interpretation}: Returns expressed as percentages (multiplied by 100), so $\sigma^2_t$ is in squared percentage points. Event coefficients $\delta_j$ represent additions to squared percentage volatility during the $[-3,+3]$ day event window.

\end{appendices}

% ============================================================================
% REFERENCES
% ============================================================================

\bibliography{references}

\end{document}
